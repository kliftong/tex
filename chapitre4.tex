\section{Comparaison et correction des estimations satellitaires}
Tang a évalu\'e les estimations de pluies  journalières de 1998 à 2007 des satellites MSWEP, TMPA, AgMERRA et PERSIANN  contre des observations au sol  sur le bassin du Mékong en  Asie du Sud-Est \cite{tang_assessing_2019}. Les résultats ont révélé que le produit MSWEP donne une bonne précision avec une erreur moyenne plus faible,  un coefficient de corrélation et une probabilité de détection de pluies rares  plus proche des valeurs idéales que les estimations PERSIANN.
 Alijanian a r\'ealis\'e ce m\^eme travail sur l'Iran avec 4 produits satellitaires (PERSIANN, CMORPH, TRMM et MSWEP) sur une période de dix ans (2003-2012) \cite{alijanian_evaluation_2017}. La performance des pluies a été évaluée à l'aide de métriques de comparaison basées sur le coefficient de corrélation,  l'erreur quadratique moyenne et l'erreur relative. L'étude montre que MSWEP a les plus hauts coefficient de corrélation  (0.72), suivis de TRMM (0.46) et de PERSIANN  (0.43) ils ont également trouv\'e que les performances des satellites  varient en fonction des régimes climatiques. Par exemple, la meilleure corrélation a été observée dans le sud, la côte du golfe Persique (un climat très chaud et humide) avec des valeurs de coefficient de corrélation de 0.72, 0.70 et 0,82 pour MSWEP, TRMM et PERSIANN respectivement. 
 Zhang quant \`a lui a repris ce travail dans une région humide de Chine  avec 5 produits satellitaires (CHIRPS, CMORPH, MSWEP, PERSIANN et TRMM) entre 1998 et 2012 \cite{zhang_suitability_2019}.
   Le coefficient de corrélation entre les précipitations mesurées et les produits satellitaires varie de 0.58 à 0.82, l’écart-type de 7.8 à 9.9 et les erreurs quadratiques moyennes de 0.52 à 0.83 mm / jour. Dans l’ensemble, le produit MSWEP donne les meilleurs résultats parmi les cinq produits satellitaires, ce qui permet d’obtenir le coefficient de corrélation le plus élevé et le plus faible erreur quadratique moyenne.  
  \\ \ \\
  Toujours sur la chine, Fang utilise les produits TRMM 3B42RT et GPM IMERG pour  \'evaluer separement les pluies maximales et les pluis rares (dépassant le 90\up{e} percentile) pour les périodes 2000-2017  \cite{fang_evaluation_2019}. Il conclut que 
  tous les jeux de données satellitaires s'accordaient bien avec les mesures au sol pour les précipitations maximales annuelles et  les précipitations rares annuelles, avec des valeurs de coefficient de corrélation élevées, supérieures à 0.72 et à 0.89 respectivement. Pour Les précipitations maximales annuelles, tous les jeux de données satellites ont présenté une légère sous-estimation tandis qu'une surestimation du volume annuel et une sous-estimation globale du taux total des précipitations rares est observée. Les performances étaient meilleures dans le sud et l'est de la Chine, que dans l'ouest avec un milieu aride à haute altitude, indiquant une influence significative de la topographie et du climat. Donc, ces résultats ont indiqué que les produits satellites présentaient un potentiel élevé pour représenter la configuration spatiale, le volume global et les caractéristiques de probabilité de précipitations rares en Chine.\\
  Sur la Guyane Française, Brochart et Andréassian ont remarqu\'e que  la corrélation des observations au sol et les \'estimations des produits TRMM et PERSIANN était assez médiocre. Ils ont développé une méthode, fonction de l'amplitude des précipitations, de la saisonnalité et de la zone considérée, permettant de corriger les estimations satellitaires afin de générer une banque de données pluviométriques longue période \cite{brochart_correction_2014}.
\section{Création d'un catalogue de tempêtes}
L'absence de mesures \`a haute r\'esolution a pouss\'e les hydrauliciens a modelis\'e la pluie comme uniforme dans l'espace. Ces mod\`eles s'av\`erent \^etre tr\`es simple d'aplication. Cependant, les incertitudes qui sont li\'ees aux hypoth\`eses simplificatices n'etaient pas pris en compte. Dot\'e des donn\'ees radar haute r\'esolution (15 mn, 1 Km\up{2}), sur une p\'eriode de 10 ans, pour une r\'egion environnant Charlotte (Caroline du Nord, États-Unis). Wright  se donne pour mission de g\'en\'erer un catalogue de temp\^etes et de reconstruire la climatologie régionale. Outre leurs   intensit\'es et leurs dur\'ees, les temp\^etes du catalogue ne seront plus supos\'ees \^etre uniforme dans l'espace mais selectionn\'ees en fonction de leurs forme et l'orientaion du bassin d'\'etude. Les pr\'ecipatations maximales annuelles obtenues seront par la suite utilis\'ees pour reconstruire la climatologie sous forme de courbe IDF (Intensi\'e-Dur\'ee-Fr\'equence) \cite{wright_estimating_2013}. L'étude a montr\'e que les courbes IDF basées sur le catalogue tendent à sous-estimer les précipitations pour de courtes périodes de retour par rapport au IDF traditionnelles, en particulier pour des courtes durées.
\section{Simulation d'inondation}
Un modèle simple d'inondation, HAND (Height Above the Nearest Drainage),  normalise la topographie par rapport au reseau de drainage, prend en entrée un modèle numérique de terrain sur lequel on injecte un volume d'eau et sort les zones qui seront susceptibles d'être inondées. Ce modèle a été valid\'e sur une surface de 18 000 km\up{2} dans le bassin versant de Rio Negro et sur d'autres bassins versants non jaugés  présentant des géologies, des géomorphologies et des types de sol différents \cite{nobre_height_2011}. 