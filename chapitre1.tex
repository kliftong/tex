
\section{Données pluviométriques}
Les données pluviométriques dont nous disposons proviennent d'une base de données gérée par le CNIGS (\emph{Centre National de l'Information G\'eo-Spacial}). Celle ci est composée de 23 stations pluviométriques r\'eparties sur tout le territoire Haïtien. Les données sont disponibles \`a partir de janvier 2013 jusqu'au mois de décembre 2018 au pas de temps journalier. Cependant, les données présentent des lacunes assez considérables. En effet, certaines stations ont des trous sur plusieurs jours, d'autres sur des mois voir des années. 

\section{Précipitations estimées \`a partir de données satellitaires}
 Les produits satellitaires de pluies reposent principalement sur l'utilisation de deux types de
mesures :
\begin{enumerate}
\item  D'une part, les images infrarouges (IR) qui fournissent une mesure de la température de brillance au sommet des nuages, qui est corrélée à la probabilité de précipitation du nuage : plus un nuage est froid et haut, plus il a de la chance de précipiter. 
\item  D'autre part, des satellites en orbite basse embarquent des capteurs micro-ondes (MW), qui absorbent les radiations émises par les nuages. Ces satellites donnent des estimations beaucoup plus fiables que les images infrarouges, mais comme il ne sont pas géostationnaires, il balaient la planète en environ une journée, et donnent donc une mesure par jour.
\end{enumerate}
La plupart des satellites combinent les deux types de mesures par différentes techniques dans le but de profiter de la qualité des données MW et de la fréquence d'échantillonnage des données IR  \cite{brochart_correction_2014}. Dans ce stage, nous utiliserons les données de 4 satellites : TRMM ; CMORPH ; PERSIANN ; MSWEP. La résolution spatio-temporelle et la période de disponibilité de ces produits sont présentées au tableau \ref{satellite}.

\subsection{TRMM}
\begin{sloppypar}
Le TRMM \emph{(Tropical Rainfall Measuring Mission)} est un programme conjoint entre la NASA (\emph{National Aeronautics and Space Administration}) et la JAXA (\emph{Japan Aerospace Exploration Agency}). C'est un algorithme d'analyse de précipitation multi-satellite qui a deux produits principaux, le TRMM 3B42RT et le TRMM 3B43RT. Le TRMM 3B42RT est un produit d'estimation de pluie quasi-temps réel, de résolution spatiale 0.25\up{°} (28 km à l'équateur) et de résolution temporelle 3 heures. Il utilise les estimations de capteurs micro-ondes (MW) quand elles sont disponibles, et les estimations infrarouge (IR) dans le cas contraire. Les estimations IR sont calibrées avec les estimations MW pour s'assurer qu'elles soient cohérentes lorsqu'elles sont simultanément disponibles \cite{brochart_correction_2014}. Les données TRMM sont disponibles \`a partir de Janvier 1998 jusqu'à date et peuvent être téléchargées via : \url{https://pmm.nasa.gov/data-access/downloads/trmm} 
\end{sloppypar}
\subsection{CMORPH}
CMORPH (\emph{CPC MORPHing technique})combine la précision des estimations en micro-onde (MW) et la résolution temporelle  des données IR. De plus, les estimations CMORPH ont été validées à l'aide de données de pluviomètre de haute qualité sur les États-Unis et en Australie et de données radar sur les États-Unis \cite{brochart_correction_2014}. Plusieurs résolutions sont disponibles : 0.25\up{0}-3h et 0.07\up{0}-30 min \cite{joyce_cmorph:_2004}. Ces données sont téléchargeables à l'adresse suivante : \url{https://www.cpc.ncep.noaa.gov/products/janowiak/cmorph_description.html} 
\subsection{PERSIANN}
PERSIANN \emph{(Precipitation Estimation from Remotely Sensed Information using Artificial Neural Networks)} a été développé par l'Université de Californie. Comme son nom l'indique, il utilise un algorithme de réseau de neurones artificiels (RNA) pour estimer le taux de précipitations en utilisant les données  IR provenant des satellites géostationnaires globaux fournies par le Centre de Prévision Climatique (CPC) et la NOAA \emph{(National Oceanic and Atmospheric Administration)}\cite{alijanian_evaluation_2017}. Néanmoins, PERSIANN utilise des données MW pour calibrer les prévisions obtenues \`a partir des images infrarouges. Sa résolution spatiale est de 0.25\up{0} et sa résolution temporelle de 3 heures. Une
version PERSIANN-CCS propose des donn\'ees \`a patir de janvier 2003 jusqu'\`a date avec une résolution spatiale de 0.04\up{0} et une r\'esolution temporelle d'une heure \cite{brochart_correction_2014}.
 Ces données sont téléchargeables à l'adresse suivante : \url{https://chrsdata.eng.uci.edu/} 
\subsection{MSWEP}
La philosophie  de MSWEP est de fusionner de manière optimale les estimations de précipitations basées sur :
\begin{enumerate}
\item les pluviomètres (WorldClim, GHCN-D, GSOD, GPCC et autres)
\item les satellites (CMORPH, GridSat, GSMaP et TMPA 3B42RT)
\item les modèles de r\'eanalyse(ERA-Interim et JRA-55)
\end{enumerate}
 
 Les biais systématiques ont \'et\'e corrig\'es \`a l'aide d'observations de débits de rivière provenant d'environ de plus d'une dizaine de milliers de stations à travers le monde \cite{beck_mswep:_2017}.\\
La version 2 (MSWEP V2) fournit des données de précipitations couvrant la période 1979-2017 avec une  résolution spatiale de 0.1\up{0} et temporelle de 3 heures \cite{beck_mswep_2018}. Les données peuvent être téléchargées via : \url{http://www.gloh2o.org/}
 
\subsection{CHIRP}
CHIRP est un ensemble de données  développé par le centre d'observation et de science des ressources de la Terre de l'USGS et le groupe des risques climatiques de l'Université de Californie à Santa Barbara gr\^ace aux  \cite{katsanos_analysis_2016} : 
\begin{enumerate}
\item  champs de précipitations obtuenus \`a partir des modèles  atmosphériques : CHPClim et CFS  de la NOAA. 
\item estimations satellitaires géostationnaires infrarouges quasi globales thermiques (TRMM 3B42 version 7)
\item  observations in situ provenant de diverses sources, y compris les services météorologiques nationaux ou régionaux
\end{enumerate}  

Les donn\'ees CHIRP ont une résolution spatiale de 0,05\up{0} et une résolution temporelle journalière et les données sont disponibles à partir de janvier 1981 jusqu'à aujourd'hui \cite{kimani_assessment_2017}.\\

\input{tableau/satellite}








