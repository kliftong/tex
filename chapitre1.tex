\section{CMORPH: une méthode qui produit des estimations des précipitations globales à partir de données passives hyperfréquences et infrarouges à haute résolution spatiale et temporelle}
Contrairement à l'IR, les signaux hyperfréquences passifs à fréquence relativement basse (PMW) (10–37 GHz) détectent l'émission thermique de gouttes de pluie, tandis que les fréquences plus élevées (85 GHz et plus) détectent la dispersion du rayonnement ascendant de la Terre vers l'espace à cause de la glace particules dans la couche de pluie et le dessus des systèmes convectifs. Toutefois, en raison des difficultés techniques qui empêchent (à ce jour) le déploiement de capteurs PMW sur des plates-formes géostationnaires, ces instruments sont limités aux plates-formes à orbite polaire; ainsi, les limites d'échantillonnage spatiales et temporelles découlant de ces observations sont sévères à moins que les données ne soient moyennées de manière substantielle dans le temps. Les estimations CMORPH ont été validées à l'aide de données de pluviomètre de haute qualité sur les États-Unis et en Australie et de données radar sur les États-Unis. \cite{joyce_cmorph:_2004}
\section{MSWEP V2 - Précipitations globales toutes les trois heures à 0,1 ° C: méthodologie et évaluation quantitative}
Nous présentons la version 2 (MSWEP V2), un ensemble de données de précipitations P quadrillées couvrant la période 1979-2017. MSWEP V2 est unique à plusieurs égards: i) une couverture mondiale complète (tous les territoires et tous les océans); ii) résolution spatiale élevée (0,1 °) et temporelle (3 heures); iii) fusion optimale de Pestimations basées sur des jauges [WorldClim, Réseau mondial de climatologie historique mondial (GHCN-D), Résumé global du jour (GSOD), Centre mondial de climatologie des précipitations (GPCC), etc.], satellites [Technique de morphing du Centre de prévision du climat (CMORPH) , Satellite quadrillé (GridSat), Cartographie globale des précipitations par satellite (GSMaP) et Mission de mesure des précipitations tropicales (TRMM), Analyse des précipitations multisatellites (TMPA) 3B42RT)], et réanalyse [Centre de prévisions météorologiques à moyen terme (CEPMM)) (ERA-Intérim) et la réanalyse japonaise de 55 ans (JRA-55)]; iv) corrections de biais de distribution, principalement pour améliorer la fréquence P ; v) correction des biais systémiques de P terrestre en utilisant le débit de la rivière Qobservations de 13 762 stations à travers le monde; vi) incorporation des observations quotidiennes de 76 747 jauges dans le monde entier; et vii) correction des différences régionales dans les délais de déclaration des jauges. MSWEP V2 se compare nettement mieux avec les données jauge – radar P de stade IV que d’autres jeux de données P de pointe pour les États-Unis, ce qui démontre l’efficacité de la méthodologie MSWEP V2. Des comparaisons globales suggèrent que MSWEP V2 présente des modèles spatiaux plus réalistes en moyenne, magnitude et fréquence. Les estimations du P moyen à long terme pour les domaines global, terrestre et océanique basées sur MSWEP V2 sont respectivement de 955, 781 et 1 025 mm an- 1 . D' autres P ensembles de données sous - estiment systématiquement Pmontants dans les régions montagneuses. En utilisant MSWEP V2, on a estimé que P se produisait 15,5\%, 12,3\% et 16,9\% du temps en moyenne pour les domaines global, terrestre et océanique, respectivement. MSWEP V2 offre des opportunités uniques d’exploration des variations spatiotemporelles de P , d’améliorer notre compréhension des processus hydrologiques et de leur paramétrage, ainsi que d’améliorer les performances des modèles hydrologiques.\cite{beck_mswep_2018}
\section{MSWEP: précipitations mondiales quadrillées à 0,25 ° (1979-2015) en combinant des données de jauge, de satellite et de réanalyse}
 Les ensembles de données sur les précipitations mondiales actuelles ( P ) ne tirent pas pleinement parti de la nature complémentaire des données satellitaires et des données de réanalyse. Nous présentons ici la version 1.1 des précipitations multi-sources pondérées par ensembles (MSWEP), un jeu de données global P pour la période 1979-2015 avec une résolution temporelle à 3 heures et une résolution spatiale de 0,25 ° spécifiquement pour la modélisation hydrologique. La philosophie de conception de MSWEP était de fusionner de manière optimale les  sources de données P de la plus haute qualité disponibles en fonction de l’échelle de temps et de l’emplacement. La moyenne à long terme de MSWEP était basée sur le jeu de données CHPclim mais a été remplacée par des jeux de données régionaux plus précis, le cas échéant. Une correction pour tenir compte de la sous-prise en jauge et des effets orographiques a été introduite en déduisant des moyennes de captage P d' après les observations de débit ( Q ) à 13762 stations à travers le monde. La variabilité temporelle du MSWEP a été déterminée par la moyenne pondérée des  anomalies de P issues de sept jeux de données. deux reposant uniquement sur l'interpolation des observations de jauge (CPC Unified et GPCC), trois sur la télédétection par satellite (CMORPH, GSMaP-MVK et TMPA 3B42RT), et deux sur la réanalyse de modèles atmosphériques (ERA-Interim et JRA-55). Pour chaque cellule de la grille, la pondération attribuée aux estimations basées sur les jauges a été calculée à partir de la densité du réseau de jauges, tandis que les pondérations attribuées aux estimations basées sur les satellites et les réanalyses ont été calculées à partir de leurs performances comparatives aux jauges environnantes. La qualité de MSWEP a été comparée à quatre  jeux de données P ajustés par jauge de dernière génération (WFDEI-CRU, GPCP-1DD, TMPA 3B42 et CPC Unified) à l’aide de Données P provenant de 125 stations tour FLUXNET dans le monde entier. MSWEP a obtenu le coefficient de corrélation quotidien le plus élevé ( R ) parmi les cinq  jeux de données P pour 60,0.\% des stations et un R médian  de 0,67 par rapport à 0,44–0,59 pour les autres jeux de données. Nous avons également évalué les performances du MSWEP en utilisant la modélisation hydrologique pour 9011 bassins versants (<50000km 2 ) à travers le monde. Plus précisément, nous avons calibré le modèle hydrologique conceptuel simple HBV (Hydrologiska Byråns Vattenbalansavdelning) par rapport aux observations Q quotidiennes  avec  P de chacun des différents jeux de données. Pour les 1058 captages peu étalonnés, représentatifs de 83,9\% de la surface terrestre mondiale (hors Antarctique), MSWEP a obtenu un étalonnage NSE médian de 0,52 contre 0,29–0,39 pour les autres  jeux de données P. MSWEP est disponible sur http://www.gloh2o.org . \cite{beck_mswep:_2017}
 \section{Correction des estimations des pluies par satellite pour les bassins versants de Guyane française}
 Les produits satellitaires de pluie reposent principalement sur l'utilisation de deux types de
mesures :
 D'une part, les images infrarouges (IR) fournissent une mesure de la température
de brillance du sommet des nuages, qui est corrélée à la probabilité de précipitation
du nuage : plus un nuage est froid, plus il est haut, et plus il a de chances de
précipiter. Cette information est assez peu fiable, mais elle a l'avantage d'être
fournie par des satellites géostationnaires, qui « voient » toujours le même disque
terrestre et peuvent le balayer toutes les demi-heures. Aujourd'hui, la constellation
des satellites météorologiques permet de couvrir toute la planète.
 D'autre part, des satellites en orbite basse embarquent des imageurs micro-ondes
(MW), absorbées par les nuages. Ils donnent des estimations beaucoup plus fiables
que les images infrarouges, mais comme il ne sont pas géostationnaires, il balaient
la planète en environ une journée, et donnent donc une mesure par jour.
En combinant les deux types de mesures par différentes techniques, on tente de profiter
de la qualité des données MW et de la fréquence d'échantillonnage des données IR.\\
Les trois produits satellitaires de pluie les plus connus sont TRMM, PERSIANN et
CMORPH.
 Le Tropical Rainfall Measuring Mission (TRMM) [1] est un programme conjoint entre
la NASA et la JAXA. C'est à la fois un satellite et une série de produits satellitaires
de pluie qui utilisent ce satellite (et d'autres satellites). TRMM 3B42RT est un
produit d'estimation de pluie quasi-temps réel, de résolution spatiale 0.25° (28 km à
l'équateur) et de résolution temporelle 3 heures. Il utilise les estimations MW quand
elles sont disponibles, et les estimations IR dans le cas contraire. Les estimations
IR sont calibrées avec les estimations MW pour s'assurer qu'elles soient
cohérentes lorsqu'elles sont simultanément disponibles.
 PERSIANN (Precipitation Estimation from Remotely Sensed Information using
Artificial Neural Networks) [2] a été développé par l'Université de Californie à Irvine
et utilise la technique du réseau de neurones pour calibrer les images IR avec les
données MW (ou avec des mesures au sol). C'est un produit qui a nécessité des
données MW pour sa calibration, mais qui en prévision n'utilise que des données
IR. Sa résolution spatiale est de 0.25° et sa résolution temporelle de 3 heures. Une
version PERSIANN-CCS propose une résolution spatiale de 0.04° (4.5 km à
l'équateur) et temporelle d'une heure.
 CMORPH (CPC MORPHing technique) [3] a été développé au Climate Prediction
Center de la NOAA. Les données MW dont utilisées comme source principale d'estimation de pluie, et sont propagées en utilisant les images IR grâce à une
technique de « morphing ». Plusieurs résolutions sont disponibles : 0.25°-3h et
0.07°-30min.\\ \ \\
Bien que les satellites observent des quantités corrélées à la pluviométrie, ils donnent des
résultats assez médiocres en pratique. Ceci s'explique par la relation ténue qui existe
entre les données IR et l'intensité des précipitations (or les données IR représentent la
majorité des données utilisées pour calculer les estimations de pluie). La figure 2
représente les estimations des pluies annuelles moyennes sur 13 ans (de 2000 à 2012)
pour TRMM et PERSIANN. On constate que la pluie TRMM est deux fois plus élevée que
la pluie PERSIANN : ce manque de consensus suffit à montrer l'importance de l'incertitude
liée aux produits satellitaires de pluie.
\\ \ \\
Pour corriger une partie des distorsions des estimations de pluie par satellite, nous avons
recherché s'il existait une erreur systématique dépendante de l'amplitude du signal. Pour
se rendre compte de l'existence d'une telle distorsion, le rapport des valeurs de pluie
pluviomètre sur les valeurs de pluie satellite, en fonction des valeurs de pluie satellite, a
été représenté sur la figure 4. Une échelle logarithmique, correspondant mieux à la
dynamique du signal de pluie, a été choisie. \cite{brochart_correction_2014}
\section{Évaluation de la climatologie des précipitations par satellite à l'aide de CMORPH, PERSIANN ‐ CDR, PERSIANN, TRMM, MSWEP sur l'Iran}
In situLes données pluviométriques observées par les jauges sont les données les plus importantes dans la gestion des ressources en eau. Cependant, ces données ont des limites spatiales et temporelles. Avec les progrès des produits de précipitations satellitaires, il est maintenant possible d'évaluer si ces produits peuvent capturer la climatologie des caractéristiques de précipitations connues. Dans cette étude, cinq estimations de précipitations satellitaires (SRE) ont été évaluées par rapport à des données de jauge basées sur différents régimes de précipitations en Iran. Les SRE évaluées sont la technique de prévision du climat au centre, l’estimation des précipitations à partir d’informations télédétectées à l’aide de réseaux de neurones artificiels (PERSIANN) et la Mission de mesure des précipitations tropicales (TRMM), l’enregistrement de données climatiques PERSIANN (PERSIANN ‐ CDR) et le plus récent système multi-sources pondéré disponible. ‐Ensemble des données de précipitation (MSWEP). La performance de ces cinq SRE est évaluée par rapport aux données de jauge (total: 958 stations) dans huit zones climatiques différentes lors de périodes journalières, mensuelles et sèches / humides sur une période de dix ans (2003-2012)\\ \ \\ 
\textbf{Tableau 2. Informations de base pour certains SRE utilisés dans cette étude.}\\ \ \\
3.2.1 Technique de morphing du centre de prévision climatique (CMORPH)
CMORPH (Joyce et al ., 2004 ) a une résolution spatiale de 0,07 ° (∼8 km) et une résolution temporelle de 0,5 heure. Dans cette technique, les vecteurs de mouvement des régimes de précipitations donnés par les satellites géostationnaires à infrarouge (IR) sont utilisés pour propager les précipitations estimées toutes les demi-heures à partir de capteurs à hyperfréquences passives (PMW). De plus, en exploitant une interpolation linéaire pondérée dans le temps, la forme et l’intensité des caractéristiques de précipitation sont transformées pendant l’intervalle séparant les passages supérieurs de capteurs PMW (Joyce et al ., 2004 ). CMORPH combine ainsi la précision de récupération supérieure des estimations en mode passif (MP) et la résolution temporelle et spatiale supérieure des données IR. Le produit final est composé uniquement de particules, les données IR étant utilisées indirectement (Dinkuet al ., 2007 ). Ces données sont téléchargeables à l’ adresse suivante : ftp://ftp.cpc.ncep.noaa.gov/precip/global_CMORPH/ .\\ \ \\
3.2.2 Mission de mesure des précipitations tropicales (TRMM)
TRMM est un algorithme d'analyse de précipitation multi-satellite qui a produit deux produits principaux, le «TRMM 3B42» et le «TRMM 3B43». Ces produits sont décrits dans Huffman et al . ( 2007). Le 3B42 est une estimation sur 3 heures des précipitations avec une résolution spatiale de 0,25 °. Cet algorithme a deux entrées. Le premier ensemble de données est constitué des données IR de satellites géostationnaires et, deuxièmement, le jeu de données à hyperfréquences passives (PMW) de plusieurs sources [imageur à micro-ondes TRMM (TMI), imageur à micro-ondes à capteur spécial (SSM / I), unité de sondage à micro-ondes avancée (AMSU) et Système d'observation radiomètre ‐ terrestre perfectionné à radiofréquence (AMSR ‐ E)]. L’estimation des précipitations par 3B42 comporte quatre étapes: (1) l’étalonnage et la combinaison d’estimations de particules, (2) l’estimation des précipitations par IR à l’aide des estimations de PM, (3) la combinaison des estimations de PM et d’IR, et (4) le rééchelonnement des données totales utilisant des observations par jauge. Ce produit est disponible depuis 1998. Dans cette étude, l'ensemble de données 3B42V7, estimation globale quotidienne à une résolution de 0,25 °, est utilisé:http://disc.gsfc.nasa.gov/uui/datasets/TRMM_3B42_Daily_V7 .\\ \ \\
3.2.3 Estimation des précipitations à partir d'informations de télédétection utilisant des réseaux de neurones artificiels (PERSIANN)
PERSIANN (Sorooshian et al ., 2000) utilise un algorithme de réseau de neurones artificiels (RNA) pour estimer le taux de précipitations en utilisant les données de température de luminosité IR provenant de satellites géostationnaires globaux fournies par le Centre de prévision climatique (CPC) et la NOAA (National Oceanic and Atmospheric Administration). Cependant, les propriétés des nuages ​​(telles que le type, la hauteur et l'épaisseur) et les conditions atmosphériques rendent très incertaines l'association des relations entre les précipitations et la température de brillance au sommet des nuages. Ainsi, les paramètres ANN sont mis à jour en utilisant les observations PMW des satellites à orbite basse [par exemple, TMI à bord du TRMM, SSM / I sur le programme de satellites météorologiques de défense, AMSR ‐ E sur les engins spatiaux Aqua et l'unité de sondage à hyperfréquence avancée ‐ B (AMSU ‐ B ) à bord de la série de satellites NOAA]. Une technique d’entraînement adaptatif (Hsu et al ., 1997) met à jour les paramètres du réseau neuronal chaque fois que des données hyperfréquences sont disponibles (AghaKouchak et al ., 2011 ). Ces données sont globalement disponibles en résolutions spatiales 0.25 ° et temporelles toutes les heures à partir de: ftp://persiann.eng.uci.edu/pub/PERSIANN/daily/ .\\ \ \\
3.2.4 Enregistrement de données climatiques PERSIANN (PERSIANN ‐ CDR)
Pour étudier les événements climatiques, il est préférable d’avoir au moins 30 ans d’enregistrement (Burroughs, 2003 ). Le développement de la performance des SER dans le but de disposer de données spatiotemporelles élevées de précipitations estimées sur une durée supérieure à 30 ans pourrait être appliqué par les recherches climatologiques. Le programme national d'enregistrement de données climatologiques (CDR) du Centre national de données climatiques de la NOAA a établi un nouvel ensemble de données rétrospectives sur les précipitations par satellite, appelé PERSIANN ‐ CDR, destiné à des études à long terme (Ashouri et al ., 2014). PERSIANN ‐ CDR est un produit de précipitation multi-satellites haute résolution fournissant des estimations des précipitations quotidiennes à une résolution spatiale de 0,25 ° de 1983 à aujourd'hui. L'algorithme de récupération utilise les données de satellites IR provenant de satellites géosynchrones mondiaux comme source principale d'informations sur les précipitations. Pour répondre aux exigences en matière d'étalonnage de PERSIANN, le modèle est pré-formé à l'aide des données de précipitations horaires de stade IV des Centres nationaux de prévision environnementale. Ensuite, les paramètres du modèle sont conservés et le modèle est exécuté pour l’enregistrement historique complet de GridSat ‐ B1 IR (Knapp, ). Les ensembles de données sont disponibles sur: www.ncdc.noaa.gov/cdr/operationalcdrs.html 2008 ). Afin de réduire les biais dans les précipitations estimées, les estimations sont ensuite ajustées à l'aide des produits de précipitation mensuels du GPCP à 2,5 ° (Ashouri et al ., 2014 .\\ \ \\
Statistiques quantitatives
Pour évaluer quantitativement la cohérence entre les SRE et les observations de jauge, le premier satellite ( S i ) et les données observées des jauges ( G i ) sont définis pour chaque pas de temps (par exemple, jour, mois, saison ou année). Les performances des SRE ont été évaluées à l'aide de mesures de comparaison basées sur le coefficient de corrélation (CC), l'erreur quadratique moyenne (RMSE), l'erreur moyenne et l'erreur relative (ER). Les formulations et les caractéristiques de ces critères peuvent être trouvées dans Hu et al . ( 2014 ) et Tan et al . ( 2015). Dans le calcul de l'ER, les données SRE moins les données de jauge sont utilisées. Une ER positive indique que l'estimation SRE est supérieure à l'observation par jauge, ce qui signifie que les quantités de précipitations sont surestimées par satellite. Une valeur négative indique que le SRE sous-estime les précipitations.\\ \ \\
4.2 Statistiques catégoriques
Les statistiques catégorielles (Katiraie ‐ Boroujerdy et al ., 2013 ; Moazami et al ., 2013 ; Mashingia et al ., 2014 ; Tan et al ., 2015 ) sont utilisées pour évaluer le potentiel des SRE pour faire la distinction entre pluvieux et nul / minuscule. conditions pluvieuses. Compte tenu du seuil de «précipitations faibles / minimes » (vitesse <1 mm jour ‐1 ), quatre conditions différentes peuvent se présenter qui seraient (Mashingia et al ., 2014).) A: hits (c'est-à-dire que l'événement devrait se produire et s'est produit), B: les fausses alarmes (c'est-à-dire que l'événement devrait se produire, mais ne s'est pas produit), C: les erreurs (c'est-à-dire que l'événement ne doit pas se produire, mais s'est produit) et D : corriger les négatifs (c'est-à-dire que l'événement ne doit pas se produire et ne s'est pas produit). Sur la base de ces définitions, les statistiques catégoriques suivantes sont évaluées. Les détails pourraient être trouvés dans Mashingia et al . ( 2014 ).

La probabilité de détection (POD) montre dans quelle mesure les SRE ont correctement détecté les précipitations pour toutes les observations de précipitations effectuées par jauges.
Le taux de fausses alarmes (FAR) mesure la fréquence à laquelle les SRE distinguent les précipitations alors qu'en réalité, elles n'étaient pas observées au sol.
L'indice de réussite critique (CSI) illustre la fraction de la précipitation d'une jauge correctement discriminée par les SRE. \cite{alijanian_evaluation_2017}
\section{Pertinence des produits de précipitation par satellite pour les simulations de bilan hydrique utilisant plusieurs observations dans un bassin hydrographique humide}
Le taux de fausses alarmes (FAR) et la probabilité de détection (POD) sont utilisés pour évaluer la précision des produits de précipitation par satellite, ce qui peut être écrit ainsi: [ 26 , 49 ]:\\
où a est le nombre d'événements de précipitations observés correctement détectés, b le nombre d'événements de précipitations détectés mais non observés et c le nombre d'événements de précipitations observés non détectés. FAR mesure la fraction de détections de précipitations qui sont des fausses alarmes, alors que POD mesure la fraction d'occurrences de précipitations correctement détectées.
Quatre indicateurs sont utilisés pour évaluer les performances de la modélisation hydrologique, à savoir le coefficient de corrélation de Pearson ( r ), le biais relatif (RB), la déviation racine-moyenne (RMSD) et le rendement de Kling-Gupta (KGE) [ 50 ]: \cite{zhang_suitability_2019}
\section{Évaluation des incertitudes de quatre produits de précipitation pour la modélisation Swat dans le bassin du Mékong}
\cite{tang_assessing_2019} comp_sat(foto)

\section{Évaluation des produits TRMM 3B42 et GPM IMERG pour l'analyse des précipitations extrêmes en Chine}
Dans cette étude, les performances de la mission de mesure des précipitations tropicales (TRMM) 3B42 et de la mesure globale des précipitationsLes données intégrées d'extraction de plusieurs satellites (GPM IMERG) dans l'estimation des précipitations extrêmes ont été évaluées sur la Chine. Les précipitations maximales annuelles et les précipitations extrêmes dépassant le 90e percentile ont été examinées et comparées aux mesures par jauge pour les périodes 2000-2017 et 2014-2017. Il a été constaté que: (1) les deux produits satellites ont capturé le schéma spatial des précipitations extrêmes bien au-dessus de la Chine, avec une sous-estimation globale du taux de précipitations extrêmes et une surestimation du volume annuel total des précipitations extrêmes; (2) Les données du TRMM 3B42 avaient une capacité limitée à détecter les événements de précipitations extrêmes, alors que GPM IMERG était légèrement supérieur; (3) les deux produits ont produit une bonne estimation des précipitations extrêmes avec des intervalles de récurrence courts-moyens, mais a présenté une sous-estimation constante à toutes les périodes de retour; (4) GPM IMERG a surperformé le TRMM 3B42 pour presque tous les paramètres d’évaluation comparés au cours de la même période; (5) les performances étaient meilleures dans le sud et l'est de la Chine avec une mousson humideclimat, que dans l'ouest de la Chine aride à haute altitude, indiquant une influence significative de la topographie et du climat. \\ \ \\
Les précipitations extrêmes ont été définies de deux manières: les maxima annuels (AM) et les pics dépassant le seuil (POT), qui étaient des événements de précipitations extrêmes avec des précipitations quotidiennes supérieures au seuil du 90e centile dans cette étude. Nous avons collecté plus de 18 ans de données TRMM 3B42 (2000-2017) et environ 4 ans de données GPM IMERG (2014-2017). En outre, afin de garantir la comparabilité dans le temps des deux produits de précipitations par satellite, la série de données TRMM de 2014 à 2017 (TRMM_2) a également été analysée en plus des séries initiales de 2000 à 2017 (TRMM_1).\\ \ \\
Plusieurs indicateurs statistiques ont été utilisés pour quantifier la cohérence entre les précipitations satellites et les données de jauge, notamment le coefficient de corrélation (R), le biais relatif (BIAS), l’erreur relative (RE) et l’erreur racine-moyenne (RMSE). Pour examiner plus en détail la capacité des données TRMM et GPM à détecter les événements pluvieux extrêmes, un ensemble de métriques catégoriques de compétences, telles que la probabilité de détection (POD), le taux de fausse alarme (FAR) et l’indice de réussite critique (CSI). POD indique la fraction d'événements de précipitations extrêmes qui ont été détectés correctement par rapport au nombre total d'événements détectés par satellite; FAR mesure la fraction de événement extrêmeles occurrences qui sont de fausses alarmes parmi le nombre total d'événements détectés par satellite; CSI désigne le rapport entre les événements de précipitations extrêmes correctement détectés par satellite et le nombre total d'événements observés ou détectés ( Chen et al., 2018 ; Wilks, 2011 ). La définition et les équations de ces indicateurs sont données dans le tableau 1\\ \ \\
\textbf{tableau comp_sat}\\ \ \\
La fréquence des précipitations extrêmes fournit des informations cruciales pour la conception des structures de protection contre les inondations et revêt une grande importance pour la gestion des risques. Il peut être estimé en utilisant des distributions de probabilité avec la théorie des valeurs extrêmes ( Coles et al., 2001 ). Deux approches sont les plus largement utilisées dans l'analyse des événements extrêmes. Les séries de maxima annuels (AM) peuvent être ajustées avec la distribution des valeurs extrêmes généralisées (GEV) à l’aide de l’approche Block Maxima; alors que l’approche des points de dépassement de seuil (POT) prend en compte tous les échantillons extrêmes dépassant un seuil haut et que la série peut être modélisée à l’aide de la distribution de Pareto généralisée (GPD) ( Acero et al., 2010 ; Katz et al., 2002 ). Les fonctions de distribution cumulative pour la distribution GEV et GPD peuvent être exprimés comme:\\ \ \\
\textbf{zo artice}\\ \ \\
\cite{fang_evaluation_2019}
\section{Évaluation de cinq produits de précipitation par satellite dans deux bassins de très faible calibre sur le plateau tibétain}
