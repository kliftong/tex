 De part sa position géographique, Haïti est sous la menace constante des cyclones qui se forment dans l'océan atlantique. Certains d'entre eux, plus violent que d'autres, ont laiss\'e derrière eux des pertes considérables. %\tiny{\url{https://www.facebook.com/notes/m\%C3\%A9t\%C3\%A9o-des-cyclones/les-ouragans-majeurs-et-les-catastrophes-naturelles-en-ha\%C3\%AFti/1211809635538112/}}.
En effet, le cyclone FLORA en octobre 1963, de catégorie 4, a tué plus de 5000 personnes et est considéré comme le sixième cyclone le plus meurtrier dans le bassin atlantique nord. Après FLORA, la seconde  tempête la plus meurtrière est JEANNE en 2004  avec un compteur de 3000 morts. Plus proche de  nous, en 2016, MATTHEW fait parti des cyclones les plus intenses et a caus\'e la mort de 1000 personnes,  des centaines de blessés et plus de 100 000 sans abri.\\ \ \\
Parallèlement,  les données hydrologiques permettant d’anticiper ces dégâts se trouvent très éparpillées parmi différentes institutions ( le Centre National de l'Information Geospacial, le Programme des Nations Unis pour le Developpement, le Comité Interministériel d’Aménagement du Territoire, les Frères de l’Instruction Chrétienne, l'Unité HydroMétéorologique d'Haïti et tant d'autres). Ce qui conduit évidemment \`a différents formats d’archivages souvent inexploitables et lorsque celles ci sont disponibles, elles ne sont pas numérisées. Cela dit, il semble difficile voir impossible de disposer d'une banque de donnée pluviométrique sur une période plus ou moins longue permettant d’anticiper les dégâts pouvant être engendrés par différents types d’événements hydrométéorologiques intenses.\\
 
Cependant, la t\'el\'edetection par satellite semble \^etre prometeuse, elle pr\'esente deux avantages principales : les donn\'ees sont graduites et elles couvrent des r\'esolutions spatiales et temporelles assez bonnes. Ainsi, elles peuvent pallier les lacunes des observations au sol en fournissant des estimations des précipitations continues à l'échelle mondiale. Bien que la précision des précipitations par satellite se soit améliorée de manière continue au cours des dernières décennies, elles souffrent toujours de sources d'erreur importantes associées aux mesures indirectes, aux algorithmes d'extraction et à la fréquence d'échantillonnage \cite{zhang_suitability_2019}. \\
\ \\
Ce travail consiste donc \`a calibrer les estimations satellitaires \`a l'aide des mesures observ\'ees au sol afin d'avoir une banque de donn\'ees pluviometriques sur une longue p\'eriode et de d\'efinir des valeurs de pluies de r\'ef\'erence  à l’échelle nationale, valeurs qui seront utiles pour le dimensionnement de structures hydrauliques et également la définition de zones inondables.\\ \ \\
\begin{sloppypar}
 Pour cela, dans un premier temps, les précipitations journali\`eres et les précipitations rares et extrêmes dépassant le 90\up{e} et le 99\up{e}  percentile respectivement de 5 produits satellitaires (CMOPRH, MSWEP, PERSIANN, TRMM et CHIRP) sseront examinées et comparées aux mesures observ\'ees au sol \`a l'aide des outils statistiques \cite{fang_evaluation_2019} pour les périodes 2013-2018. Les estimations satellitaires seront ensuite corrig\'ee en se basant sur la m\'ethode utilis\'ee par Brochard et Adressian  \cite{brochart_correction_2014}.
 \end{sloppypar}
  \\ \ \\
Les données ainsi disponibles, la prochaine étape consiste \`a reconstuire la climatologie locale.
En effet, Le plus gros soucis de l'aménageur hydraulique est de savoir quelle pluie provoquera l'inondation de son ouvrage, dans le meilleur de cas, de trouver également sa fréquence d'apparition. Pour résoudre son problème, l'aménageur, dépendant des moyens économiques disponibles, suppose une pluie (durée et intensité) et il cherche la fréquence de dépassement de sa pluie de conception. Ce procédé a donné naissance aux courbes IDF (Intensité-Durée-Fréquence) qui peuvent être obtenues \`a l'aide des données pluviométriques enregistr\'ees sur une période de temps plus ou moins longue.\\
Les courbes IDF ont eu un large succès notamment chez ceux qui évaluent les risques d'inondations. N\'eanmoins, l'observation montre que les inondations ne d\'ependent pas uniquement de la dur\'ee et de l'intensit\'e des temp\^etes mais \'egalement de sa taille, de sa forme et de sa vitesse. Justement, une courte pluie mais tr\`es intense r\'epond diff\'eremment suivant que le bassin d'\'etude est \'etroit ou assez vaste. En effet , celle ci peut constituer une menace  d'inondation pour un petit bassin, mais peut ne pas représenter un danger pour un réseau hydrographique plus vaste. De m\^eme, une longue pluie  et peu intense  pourrait entra\^iner des inondations sur un fleuve important en raison de l'accumulation progressive d'eau dans les sols, les chenaux des cours d'eau et les réservoirs, mais ne peut jamais  provoquer une inondation subite à petites échelles. Ceci \'etant dit, la structure spatio-temporelle des précipitations et son importance doivent donc être prise en compte \cite{wright_remote_2017}.\\ \ \\
La pluviométrie ne sera donc plus considérée comme dépendante de  deux paramètres mais de trois : l'intensité, la durée et la structure spatio-temporelle.  Les deux premières composantes (la durée et l'intensité) font l'objet de recherche et d'application depuis des décennies, et très utilis\'ee dans les bureaux d'études sous le vocabulaire de courbes IDF, déterminent l'ensemble de la distribution de probabilité du volume des précipitations  en un point ou dans une zone. La troisième composante, la structure espace-temps, décrit la variabilité spatiale et temporelle des précipitations et est déterminée par la taille de la tempête, la vitesse et l'évolution temporelle de celle ci. La structure espace-temps peut donc être comprise comme décrivant le \emph{quand} et le \emph{o\`u} des précipitations extrêmes, alors que l'intensité et la durée décrivent le \emph{combien} \cite{wright_remote_2017}. Une fois le catalogue de tempêtes généré, les estimations de courbes Intensité-Durée-Fréquence (IDF) pour Haïti seront déduites \cite{wright_estimating_2013}. \\ \ \\


Une fois les courbes IDF construites, nous utiliserons le modèle HAND (\emph{Height Above Nearest Drainage} \cite{nobre_height_2011}) qui, dans un premier temps, génère un réseau hydrographique en tenant compte des incertitudes liées aux dépressions dans les MNT (\emph{Modèle Numérique de Terrain}) traditionnels. Ensuite, pour chaque maille du réseau hydrographique, le modèle normalise les altitudes des mailles de sa surface tributaire  par rapport \`a l'altitude de celle ci. Le réseau hydrographique est maintenant converti en une référence topographique normalisée, de sorte que le modèle HAND ne conserve plus de référence au niveau de la mer. Ainsi,nous pourrons estimer les emprises d’inondations par débordement du réseau de drainage
à partir d’une certaine hauteur de débordement. L’analyse des emprises d’inondations
générée par différentes hauteurs de débordement permettra de déterminer des seuils critiques
de cumul de pluie.
